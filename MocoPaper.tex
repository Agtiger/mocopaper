% Template for PLoS
% Version 3.5 March 2018
%
% % % % % % % % % % % % % % % % % % % % % %
%
% -- IMPORTANT NOTE
%
% This template contains comments intended 
% to minimize problems and delays during our production 
% process. Please follow the template instructions
% whenever possible.
%
% % % % % % % % % % % % % % % % % % % % % % % 
%
% Once your paper is accepted for publication, 
% PLEASE REMOVE ALL TRACKED CHANGES in this file 
% and leave only the final text of your manuscript. 
% PLOS recommends the use of latexdiff to track changes during review, as this will help to maintain a clean tex file.
% Visit https://www.ctan.org/pkg/latexdiff?lang=en for info or contact us at latex@plos.org.
%
%
% There are no restrictions on package use within the LaTeX files except that 
% no packages listed in the template may be deleted.
%
% Please do not include colors or graphics in the text.
%
% The manuscript LaTeX source should be contained within a single file (do not use \input, \externaldocument, or similar commands).
%
% % % % % % % % % % % % % % % % % % % % % % %
%
% -- FIGURES AND TABLES
%
% Please include tables/figure captions directly after the paragraph where they are first cited in the text.
%
% DO NOT INCLUDE GRAPHICS IN YOUR MANUSCRIPT
% - Figures should be uploaded separately from your manuscript file. 
% - Figures generated using LaTeX should be extracted and removed from the PDF before submission. 
% - Figures containing multiple panels/subfigures must be combined into one image file before submission.
% For figure citations, please use "Fig" instead of "Figure".
% See http://journals.plos.org/plosone/s/figures for PLOS figure guidelines.
%
% Tables should be cell-based and may not contain:
% - spacing/line breaks within cells to alter layout or alignment
% - do not nest tabular environments (no tabular environments within tabular environments)
% - no graphics or colored text (cell background color/shading OK)
% See http://journals.plos.org/plosone/s/tables for table guidelines.
%
% For tables that exceed the width of the text column, use the adjustwidth environment as illustrated in the example table in text below.
%
% % % % % % % % % % % % % % % % % % % % % % % %
%
% -- EQUATIONS, MATH SYMBOLS, SUBSCRIPTS, AND SUPERSCRIPTS
%
% IMPORTANT
% Below are a few tips to help format your equations and other special characters according to our specifications. For more tips to help reduce the possibility of formatting errors during conversion, please see our LaTeX guidelines at http://journals.plos.org/plosone/s/latex
%
% For inline equations, please be sure to include all portions of an equation in the math environment.  For example, x$^2$ is incorrect; this should be formatted as $x^2$ (or $\mathrm{x}^2$ if the romanized font is desired).
%
% Do not include text that is not math in the math environment. For example, CO2 should be written as CO\textsubscript{2} instead of CO$_2$.
%
% Please add line breaks to long display equations when possible in order to fit size of the column. 
%
% For inline equations, please do not include punctuation (commas, etc) within the math environment unless this is part of the equation.
%
% When adding superscript or subscripts outside of brackets/braces, please group using {}.  For example, change "[U(D,E,\gamma)]^2" to "{[U(D,E,\gamma)]}^2". 
%
% Do not use \cal for caligraphic font.  Instead, use \mathcal{}
%
% % % % % % % % % % % % % % % % % % % % % % % % 
%
% Please contact latex@plos.org with any questions.
%
% % % % % % % % % % % % % % % % % % % % % % % %

\documentclass[10pt,letterpaper]{article}
\usepackage[top=0.85in,left=2.75in,footskip=0.75in]{geometry}

% amsmath and amssymb packages, useful for mathematical formulas and symbols
\usepackage{amsmath,amssymb}

% Use adjustwidth environment to exceed column width (see example table in text)
\usepackage{changepage}

% Use Unicode characters when possible
\usepackage[utf8x]{inputenc}

% textcomp package and marvosym package for additional characters
\usepackage{textcomp,marvosym}

% cite package, to clean up citations in the main text. Do not remove.
\usepackage{cite}

% Use nameref to cite supporting information files (see Supporting Information section for more info)
\usepackage{nameref,hyperref}

% line numbers
\usepackage[right]{lineno}

% ligatures disabled
\usepackage{microtype}
\DisableLigatures[f]{encoding = *, family = * }

% color can be used to apply background shading to table cells only
\usepackage[table]{xcolor}

% array package and thick rules for tables
\usepackage{array}

% create "+" rule type for thick vertical lines
\newcolumntype{+}{!{\vrule width 2pt}}

% create \thickcline for thick horizontal lines of variable length
\newlength\savedwidth
\newcommand\thickcline[1]{%
  \noalign{\global\savedwidth\arrayrulewidth\global\arrayrulewidth 2pt}%
  \cline{#1}%
  \noalign{\vskip\arrayrulewidth}%
  \noalign{\global\arrayrulewidth\savedwidth}%
}

% \thickhline command for thick horizontal lines that span the table
\newcommand\thickhline{\noalign{\global\savedwidth\arrayrulewidth\global\arrayrulewidth 2pt}%
\hline
\noalign{\global\arrayrulewidth\savedwidth}}


% Remove comment for double spacing
%\usepackage{setspace} 
%\doublespacing

% Text layout
\raggedright
\setlength{\parindent}{0.5cm}
\textwidth 5.25in 
\textheight 8.75in

% Bold the 'Figure #' in the caption and separate it from the title/caption with a period
% Captions will be left justified
\usepackage[aboveskip=1pt,labelfont=bf,labelsep=period,justification=raggedright,singlelinecheck=off]{caption}
\renewcommand{\figurename}{Fig}

% Use the PLoS provided BiBTeX style
\bibliographystyle{plos2015}

% Remove brackets from numbering in List of References
\makeatletter
\renewcommand{\@biblabel}[1]{\quad#1.}
\makeatother



% Header and Footer with logo
\usepackage{lastpage,fancyhdr,graphicx}
\usepackage{epstopdf}
%\pagestyle{myheadings}
\pagestyle{fancy}
\fancyhf{}
%\setlength{\headheight}{27.023pt}
%\lhead{\includegraphics[width=2.0in]{PLOS-submission.eps}}
\rfoot{\thepage/\pageref{LastPage}}
\renewcommand{\headrulewidth}{0pt}
\renewcommand{\footrule}{\hrule height 2pt \vspace{2mm}}
\fancyheadoffset[L]{2.25in}
\fancyfootoffset[L]{2.25in}
\lfoot{\today}

%% Include all macros below

\newcommand{\lorem}{{\bf LOREM}}
\newcommand{\ipsum}{{\bf IPSUM}}
\newcommand{\analytic}{
\begin{equation}
    \begin{alignat*}{2}
        \mbox{minimize}
         \quad & \int_{t_0}^{t_f} \frac{1}{2}x^2~dt &&  \\
         \mbox{subject to}
         \quad & \dot{q} = u \\
         & \dot{u} = -u + x \\
         & q_0 = 0 \\
         & u_0 = 0 \\
         & q_f = 5 \\
         & u_f = 2
    \end{alignat*}
\end{equation}
}

%% END MACROS SECTION


\begin{document}
\vspace*{0.2in}

% Title must be 250 characters or less.
\begin{flushleft}
{\Large
\textbf\newline{Title of submission to PLOS journals} % Please use "sentence case" for title and headings (capitalize only the first word in a title (or heading), the first word in a subtitle (or subheading), and any proper nouns).
}
\newline
% Insert author names, affiliations and corresponding author email (do not include titles, positions, or degrees).
\\
Name1 Surname\textsuperscript{1,2\Yinyang},
Name2 Surname\textsuperscript{2\Yinyang},
Name3 Surname\textsuperscript{2,3\textcurrency},
Name4 Surname\textsuperscript{2},
Name5 Surname\textsuperscript{2\ddag},
Name6 Surname\textsuperscript{2\ddag},
Name7 Surname\textsuperscript{1,2,3*},
with the Lorem Ipsum Consortium\textsuperscript{\textpilcrow}
\\
\bigskip
\textbf{1} Affiliation Dept/Program/Center, Institution Name, City, State, Country
\\
\textbf{2} Affiliation Dept/Program/Center, Institution Name, City, State, Country
\\
\textbf{3} Affiliation Dept/Program/Center, Institution Name, City, State, Country
\\
\bigskip

% Insert additional author notes using the symbols described below. Insert symbol callouts after author names as necessary.
% 
% Remove or comment out the author notes below if they aren't used.
%
% Primary Equal Contribution Note
\Yinyang These authors contributed equally to this work.

% Additional Equal Contribution Note
% Also use this double-dagger symbol for special authorship notes, such as senior authorship.
\ddag These authors also contributed equally to this work.

% Current address notes
\textcurrency Current Address: Dept/Program/Center, Institution Name, City, State, Country % change symbol to "\textcurrency a" if more than one current address note
% \textcurrency b Insert second current address 
% \textcurrency c Insert third current address

% Deceased author note
\dag Deceased

% Group/Consortium Author Note
\textpilcrow Membership list can be found in the Acknowledgments section.

% Use the asterisk to denote corresponding authorship and provide email address in note below.
* correspondingauthor@institute.edu

\end{flushleft}
% Please keep the abstract below 300 words
\section*{Abstract}
Lorem ipsum dolor sit amet, consectetur adipiscing elit. Curabitur eget porta erat. Morbi consectetur est vel gravida pretium. Suspendisse ut dui eu ante cursus gravida non sed sem. Nullam sapien tellus, commodo id velit id, eleifend volutpat quam. Phasellus mauris velit, dapibus finibus elementum vel, pulvinar non tellus. Nunc pellentesque pretium diam, quis maximus dolor faucibus id. Nunc convallis sodales ante, ut ullamcorper est egestas vitae. Nam sit amet enim ultrices, ultrices elit pulvinar, volutpat risus.


% Please keep the Author Summary between 150 and 200 words
% Use first person. PLOS ONE authors please skip this step. 
% Author Summary not valid for PLOS ONE submissions.   
\section*{Author summary}
Lorem ipsum dolor sit amet, consectetur adipiscing elit. Curabitur eget porta erat. Morbi consectetur est vel gravida pretium. Suspendisse ut dui eu ante cursus gravida non sed sem. Nullam sapien tellus, commodo id velit id, eleifend volutpat quam. Phasellus mauris velit, dapibus finibus elementum vel, pulvinar non tellus. Nunc pellentesque pretium diam, quis maximus dolor faucibus id. Nunc convallis sodales ante, ut ullamcorper est egestas vitae. Nam sit amet enim ultrices, ultrices elit pulvinar, volutpat risus.

\linenumbers

% Use "Eq" instead of "Equation" for equation citations.
\section*{Introduction}

OpenSim Moco: musculoskeletal optimal control

Journal: PLOS Computational Biology (Software Section; this journal gives special support for pre-prints on bioRxiv).

https://journals.plos.org/ploscompbiol/s/presubmission-inquiries


Nick’s 3D walking tracking w/contact and MocoInverse
Your 2D prediction walking
3D tracking with contact and contact parameter optiization?
		-> difficult with a Simbody force. Possible but takes more work.
w/ parameter opt for device
Compare MocoInverse to CMC activations and runtime
Mention that we use a different muscle model?


Abstract
OpenSim Moco solves optimal control problems with OpenSim musculoskeletal models using the direct collocation method.
The direct collocation method is often faster or more flexible than other methods popular in biomechanics, such as Static Optimization, Computed Muscle Control, and single shooting (for predicting movements). But direct collocation requires expertise to implement efficiently. Moco frees researchers from implementing direct collocation themselves, allowing them to focus on their scientific questions.
Moco can solve the wide range of problems that interest movement biomechanists: tracking motion, predicting motion, muscle parameter optimization, model fitting (residual reduction), dynamically-consistent inverse kinematics, EMG-driven simulation, device design, and others.
TODO summarize results
Moco is the first musculoskeletal direct collocation tool to seamlessly handle kinematic constraints, which are common in OpenSim models.
We designed Moco to be an easy-to-use, flexible, extensible, and verified community resource, all of which we find essential for using simulation to improve mobility.

Introduction

Musculoskeletal simulations have improved human health and mobility by discovering ways to walk that reduce knee pain~\cite{Fregly:2009}, by revealing that children with cerebral palsy exhibit simplified motor control when walking~\cite{Steele:2015}, and by reproducing eye movement disorders~\cite{Priamikov:2016}. Simulations help keep us safe as we push our limits, whether that’s designing exercise equipment for preserving bone density in outer space~\cite{Fregly:2015}, designing exoskeletons to reduce back injuries from heavy lifting~\cite{Manns:2016}, or studying knee ligament injuries in soccer~\cite{Thompson:2017}.

Biomechanists continue to be more ambitious with the questions they pursue, but our current tools are not up to the task. OpenSim is a popular musculoskeletal modeling and simulation package~\cite{Delp:2007ij,Seth:2011hya,Seth:2018gg} that provides tools for fitting certain model parameters to match experimental motion data, but these tools—Scale and Residual Reduction Algorithm—do not permit users to fit arbitrary parameters in a model. OpenSim’s Static Optimization and Computed Muscle Control~\cite{Thelen:2003bba} tools solve for muscle forces that track an observed motion while minimizing an objective such as the sum of squared normalized muscle forces. However, neither tool optimizes the controls over the movement’s entire duration, and neither allows altering the objective function. Some of the most interesting biomechanics questions require predicting a motion rather than tracking data (e.g., solving for a gait that minimizes energy consumption). The field has accurately predicted healthy and pathological gaits, but the common methods for these predictions are limited to simple models and few subjects because the methods require 10 or more hours to converge~\cite{Dorn:2015ji,Song:2018ji,Ong:2019TODO}. All these problems are naturally posed as optimal control problems: we seek a system’s parameters and time-dependent states and controls that minimize a cost (e.g., effort) subject to the dynamics of the system expressed as differential-algebraic equations (DAEs). The ad-hoc methods currently used for musculoskeletal optimal control problems limit progress in the field.

A growing method for solving optimal control problems is to approximate the system’s states and controls as polynomial splines and solve for the knot points that lead the spline to obey the system’s dynamics. The dynamics are enforced by requiring the time derivative of the state splines to match the derivative from the system’s differential equations at certain time points. This method is called “direct collocation”, because the spline derivatives are “collocated” with the exact derivatives; see page 211 in~\cite{Hairer:1993} and page 498 in~\cite{Hairer:1996}. Direct collocation is useful for musculoskeletal problems because it is flexible and fast: the method can handle arbitrary objectives, can optimize model parameters, and leads to sparse nonlinear programs that can be readily solved by generic optimization software.

The advantages of direct collocation have led biomechanists to evaluate the method~\cite{Betts:2010,Umberger:2018ec,Mombaur:2016eb,Kelly:2017} for tracking kinematics~\cite{Lin:2017jp}, predicting kinematics~\cite{Porsa:2015dn,Meyer:2016gl,Lee:2016dn,KMoore:2018ea,Lin:2018ex}, and fitting muscle properties~\cite{Falisse:2016}. We now know how to efficiently handle multibody and muscle dynamics~\cite{vandenBogert:2011fv,Groote:2016dq}, solve the muscle redundancy problem~\cite{Groote:2016dq}, include energy consumption in the objective~\cite{Koelewijn:2018kw,Koelewijn:2019}, and employ automatic differentiation to rapidly simulate complex muscles~\cite{Falisse:2019a}. Using these methodological advances, we have learned that minimizing an energy-related cost produces non-physiological knee flexion during walking~\cite{Ackermann:2010dd}, that rearfoot striking is more energy efficient than forefoot striking~\cite{Miller:2015fc}, that skipping is the most efficient gait on our moon~\cite{Ackermann:2012}, that the energetically optimal fiber type to recruit depends on the demands of a motion~\cite{Lai:2018} and that unilateral amputees can improve gait symmetry with only a minor increase in effort~\cite{Koelewijn:2016bm}. TODO, Falisse2019b Nguyen2019 Mehrabi2019 Rohani2017 (assistive device)

Most biomechanists prefer to focus on answering compelling scientific questions and avoid writing complex software for direct collocation, which requires careful bookkeeping of variables and constraint equations and efficiently computing sparse finite differences (for derivative-based optimization). Many biomechanists have graciously shared their code, but such code either is tailored to specific models or motions, handles only unconstrained models, contains closed-source components, is inefficient, or is difficult to install on other computers. Lastly, choosing the problem formulation (e.g., expressing dynamics as explicit or implicit differential equations) and solver settings (e.g., first-order or second-order Newton methods) that lead to fast convergence requires expertise; ideally, such expertise is embedded into the software via reasonable defaults, and users can edit their formulation or solver settings by editing single lines of code.

To improve the accessibility of advanced optimal control methods in musculoskeletal biomechanics, we introduce OpenSim Moco: an easy-to-use, flexible, and extensible community resource for solving optimal control problems with OpenSim models. Just as OpenSim frees biomechanists from implementing equations of motion on their own, OpenSim Moco frees biomechanists from implementing optimal control methods. In this paper, we answer scientifically relevant questions with Moco by solving four types of problems: motion prediction, motion tracking, muscle redundancy, and parameter optimization. In the process, we verify that tracking a previously predicted motion yields the original muscle behavior, and we validate that simulated muscle activity and motion match experimental data. We examine both whole-body weight lifting and upper-body reaching, and design devices that reduce the human effort required in these activities.

Design and Implementation

Moco provides two interfaces: a simple “tool” interface for common problems and an advanced “study” interface for custom problems. The “tool” interface uses the “study” interface internally, so we begin by describing the “study” interface.


Fig 1: Moco is flexible but provides convenient interfaces for common problems.

The “study” interface (sub-section)
The “study” interface starts with the MocoStudy class, which includes a MocoProblem and a MocoSolver (Fig 1). We denote names of classes in Moco with italics, and all these classes are available via C++, Matlab, Python, and XML text files, with interfaces familiar to OpenSim users.

What problems can Moco solve? (sub-sub-section)

We designed MocoProblem to support the features that researchers find important for answering their scientific questions.
cost terms (MocoGoal): Users can minimize a mix of control effort, error from an observed motion, and joint reaction loads.
endpoint constraints (also MocoGoal): Users can enforce symmetry or periodicity with constraints relating initial and final states.
multibody dynamics, auxiliary (muscle) dynamics, and kinematic constraints (Model): OpenSim Models are a standard way to describe musculoskeletal systems, and Moco uses OpenSim Models to obtain the system’s differential equations and kinematic constraints. Kinematic constraints are commonly used to model anatomy such as the patella, the medial and lateral compartments of the knee, the clavicle, and the neck~\cite{Seth:2016,Lerner:2015,Rajagopal:2016ek,Cazzola:2017}; so we prioritized support for kinematic constraints.
algebraic path constraints (MocoPathConstraint): Researchers often estimate muscle activity with electromyography, and Moco allows constraining simulated muscle excitation to be close to those measurements~\cite{Pizzolato:2015,Falisse:2016}.
parameter optimization (MocoParameter): Users can optimize static model properties such as a body’s mass, a muscle’s optimal fiber length, or an exoskeleton’s stiffness.
Cost terms and endpoint constraints are both expressed as MocoGoals, as they have a similar structure: they depend on the initial and final states and controls and on an integral over the motion. Additionally, MocoProblem allows bounds on states, controls, and initial and final time.

Mathematically, a MocoProblem describes the following optimization problem:


In this optimization problem, we seek the time-dependent states, $y(t)$, and controls, $x(t)$, that minimize a sum of cost functionals $J_j$ with weights $w_j$. The states include generalized coordinates, $q(t)$, generalized speeds, $u(t)$, and auxiliary states, $z(t)$, such as muscle activation and fiber length. We may also seek static (time-invariant) parameters, $p$, the initial time of the motion $t_i$, or the final time of the motion $t_f$. We place lower and upper bounds (denoted by subscripts L and U, respectively) on all variables, as well as bounds on the initial and final values of the states and controls.

The variables must obey the system’s multibody dynamics (involving the mass matrix, $M$; applied forces, $f_\mathrm{app}$, from gravity, muscles, etc.; and centripetal and Coriolis terms, $f_\mathrm{bias}$) and auxiliary dynamics, $f_\mathrm{aux}$. The system may contain kinematic constraints at the position level (e.g., to prescribe the location of the patella based on the knee angle), $\phi$, velocity level, $\nu$, or acceleration level, $\alpha$.When kinematic constraints exist, we introduce time-dependent Lagrange multiplier variables, $\lambda$, to enforce the constraints with forces applied in directions determined by the kinematic constraint Jacobian, $G$.

Additionally, the variables must obey algebraic (non-differential) path constraints, g, over the motion, with static bounds $g_L$ and $g_U$; and endpoint constraints $V_k$, with static bounds $V_{L,k}$ and $V_{U,k}$. The cost terms and endpoint constraints may depend on initial and final time, states, controls, kinematic constraint multipliers (required for joint reactions), static parameters, and an integral, $S_\mathrm{c,j}$ or $S_\mathrm{e,k}$, over the motion.

Fig 1 shows the goals and path constraints that Moco provides. Users wishing to employ a cost term, endpoint constraint, or path constraint that Moco does not yet support can create a C++ plugin using the same steps as for OpenSim plugins.

By providing a library of cost, endpoint constraint, and path constraint modules, allowing users to create their own modules, and allowing these modules to be combined, we achieve our design goals of ease-of-use, flexibility, and extensibility.

How does Moco solve problems? (sub-sub-section)

All details of how Moco solves an optimal control problem are encapsulated in MocoSolver. Solving a musculoskeletal optimal control problem often requires trying many problem formulations and solver settings, such as expressing dynamics as explicit or implicit differential equations, using a first-order or second-order optimization algorithm, and using a trapezoidal or Hermite–Simpson transcription scheme~\cite{Betts:2010}. With Moco, users can change each of these settings with a single line of code, allowing users to focus on their research question.

The MocoProblem and MocoSolver are decoupled to afford the user with flexibility. The MocoProblem knows nothing about the solver that may be used. When a user defines a custom cost term, they need not worry about how the solver will handle the cost. The only assumption made by MocoSolver about the problem is that it describes a multibody system; this allows using special solver algorithms not suited to generic dynamic systems, such as for handling kinematic constraints~\cite{Posa:2015}. Moco users can add bodies or muscles to their model and use the same exact solver, whereas homebrewed code may require updating the direct collocation algorithm.

Moco provides two direct collocation solvers that transcribe continuous optimal control problems into finite-dimensional nonlinear programs (NLPs) that are passed on to well-established derivative-based NLP solvers. MocoCasADiSolver uses the third-party CasADi library~\cite{Andersson:2019}, and MocoTropterSolver uses a direct collocation solver we developed called Tropter. CasADi is an open-source package for algorithmic differentiation and is a bridge to NLP solvers IPOPT~\cite{Wachter:2006}, SNOPT~\cite{Gill:2005}, and others.



MocoCasADiSolver
MocoTropterSolver
dependencies
CasADi, IPOPT, SNOPT (opt)
Tropter, Eigen, ColPack, ADOL-C (opt), IPOPT, SNOPT (opt)
supported NLP solvers
IPOPT, SNOPT, and others.
IPOPT, SNOPT
parallelized
yes
no
transcription schemes
trapezoidal and Hermite–Simpson
trapezoidal and Hermite–Simpson
dynamics modes
implicit and explicit
implicit (with no kinematic constraints) and explicit
automatic differentiation
symbolic algorithmic differentiation
operator overloading with ADOL-C
finite differences
Adaptive step size (TODO)
Second-order [Bohme and Frank], TODO
parameter optimization
fast if initSystem() is not required
fast
least permissive license (including IPOPT but no other nonlinear program solvers)
GNU Lesser General Public License 3.0 (weak copyleft)
Eclipse Public License, Mozilla Public License

Derivative-based nonlinear program solvers require the gradient of the objective, the Jacobian of the constraints, and sometimes the Hessian of the Lagrangian~\cite{Betts:2010}. To maximize computational efficiency, these derivatives are ideally computed exactly through either analytic equations or automatic differentiation~\cite{Andersson:2019,Walther:2003}. OpenSim’s main distribution does not provide exact derivatives, so we resort to finite differences. CasADi is an ideal library for employing direct collocation, but two limitations led us to create Tropter: CasADi did not initially support finite differences, and CasADi has a more restrictive open-source license. More recent versions of CasADi support finite differences and CasADi understands the structure of the objective and constraint functions, allowing for potentially more efficient finite difference calculations than with Tropter, which treats the objective and constraints as black-box functions~\cite{Patterson:2012}. If OpenSim provides exact derivatives in the future, we can exploit Tropter’s and CasADi’s automatic differentiation modes~\cite{Falisse:2019a}.

Those distributing Moco as a dependency of closed-source software may prefer distributing Moco without CasADi, as CasADi’s “weak copyleft” GNU Lesser General Public License (3.0) places certain requirements on how CasADi software is redistributed.

Both the Tropter and CasADi solvers provide two transcription schemes: the second-order trapezoidal scheme and the third-order Hermite–Simpson scheme [cite Betts 2010]. Multibody dynamics can be expressed with either explicit differential equations (“forward dynamics”) or implicit differential equations (“inverse dynamics”); the implicit mode may lead to a more robust problem~\cite{vandenBogert:2011fv}, especially for systems with light bodies. These solvers currently handle only position-level (non-holonomic) constraints $\phi$; velocity-level (holonomic) constraints $\nu$ and acceleration-level constraints $\alpha$ are not supported yet.

Trapezoidal transcription

As a second-order scheme, trapezoidal transcription exhibits accuracy that improves four-fold when halving the mesh interval (i.e., time step).

We discretize the continuous variables $t$, $y$, $x$, and $\lambda$ on a mesh of time points $t_i$ defined by dimensionless time $\tau_i$, yielding mesh intervals $h_i$:



For conciseness, we define the following functions:


where $\mathrm{trap}_i()$ is a trapezoidal rule approximation to an integral for mesh interval $i$, and $\eta$ represents any subset of continuous variables.

The result of the trapezoidal transcription, with multibody dynamics expressed as explicit differential equations, is the following NLP:



We integrate the integral cost terms and the differential equations using the trapezoidal rule. Algebraic constraints—those involving $\phi$ and $g$—are enforced only at the mesh points; the quadratic spline approximation of the states may violate the constraints within the mesh intervals.

When expressing the multibody dynamics implicitly, we remove the constraint involving $f_\mathrm{mb}$, introduce generalized accelerations as an algebraic (control) variable $\xi$, and enforce multibody dynamics in “inverse dynamics” form.

The constant $w_B$ is a large positive number (1000 by default).

Our implementation of trapezoidal transcription handles kinematic constraints, but not in the most robust fashion. We constrain $\phi$ but not its time derivatives; this results in an index-3 differential-algebraic equation system, which is challenging to solve~\cite{Hairer:1996} (TODO: correspondence with van den Bogert). Furthermore, multiple values of the constraint multipliers $\lambda$ could satisfy the constraints, so we minimize these multipliers (with weight $w_\lambda$) to ensure their uniqueness.


Hermite–Simpson transcription

TODO

Postprocessing a solution

To solve a problem, users use the solve() function of the MocoStudy class, which provides the user with a MocoSolution object. The MocoSolution class derives from MocoTrajectory, which provides easy access to the values of all variables at any iteration in the optimization algorithm. Users provide initial guesses via a MocoTrajectory, and can use the solution object from one problem as the initial guess for a subsequent problem; this permits users to build a complex problem by solving a series of simpler problems. MocoSolution provides additional information, including whether the solver converged, the final objective value, and the number of solver iterations. After solving a problem, users often wish to visualize the solution as an animation, plot the state and control trajectories, or compute quantities from the solution. With a MocoStudy and MocoSolution, each of these tasks require only a single line of code. MocoTrajectories can be written to and read from tab-delimited Storage text files, which are familiar to OpenSim users. The ability to save MocoTrajectories and MocoStudies to files is valuable for reproducing results~\cite{Peng:2011}.

Prescribed kinematics

A common goal in musculoskeletal biomechanics is to estimate muscle and actuator behavior that drove an observed motion. We can solve this problem by minimizing the error between the observed motion and the simulated motion, as with Computed Muscle Control (using the "slow target") or MocoTrack (described later). Alternatively, we can prescribe the motion exactly, as with Static Optimization, EMG-driven simulation~\cite{Lloyd:2003}, and the Muscle Redundancy Solver~\cite{Groote:2016dq}. Prescribing kinematics leads to a problem that is robust and fast, as the nonlinear multibody dynamics are no longer part of the optimization problem, but prevents predicting deviations from the observed motion.

When we prescribe kinematics, we obtain the following optimal control problem:

In this formulation, the kinematic variables q and u are replaced with known quantities $\hat{q}$ and $\hat{u}$. The system still contains auxiliary state variables z, control variables x, and auxiliary dynamics. If none of the parameter variables affect the multibody system, then the multibody dynamics are reduced to a force balance: applied forces must match the net generalized forces determined by the kinematics. Users can prescribe kinematics using the PositionMotion model component in Moco. Note that this type of prescribed motion—which removes degrees of freedom—differs from the prescribed motion in OpenSim’s Coordinate, which adds kinematic constraints to the system.

The “tool” interface (sub-section)

While the “study” interface is powerful, many questions can be answered by more standard tools with simpler interfaces, similar to OpenSim’s existing tools like the Scale Tool and Computed Muscle Control. Currently, Moco provides two tools: MocoTrack solves motion tracking problems, wherein the system tracks motion data while minimizing a desired metric; and MocoInverse solves the muscle/actuator redundancy problem, wherein the system’s motion is prescribed (using PositionMotion) and we seek muscle (or other actuator) controls that achieve the motion (Fig 1; similar to Computed Muscle Control). In both cases, the only required inputs are an OpenSim model and motion data (marker trajectories and external forces). Internally, both tools use the MocoStudy interface with solver settings that we found to yield good performance.

Verification (sub-section)

\analytic

TODO

Results



Availability

OpenSim Moco can be downloaded freely for Windows and Mac from https://github.com/opensim-org/opensim-moco, where we develop the project and users can report bugs and request features. The OpenSim Moco source code is available under the permissive Apache License 2.0, though some dependencies have more restrictive licenses (e.g., CasADi is available under the “weak copyleft” GNU Lesser General Public License).

The documentation is available online at https://opensim-org.github.io/opensim-moco and within the OpenSim Moco distribution, and contains a User Guide, Theory Guide, Developer Guide, and an Application Programming Interface (API) Reference. The User Guide explains how to use Moco and provides tips for posing a problem. Moco’s simple interface makes it an ideal resource for teaching optimal control methods in a biomechanics course; to this end, the User Guide contains a section on using Moco in a course with lecture slides and guided hands-on examples (TODO: Add this). The Theory Guide explains how Moco implements direct collocation, and the Developer Guide introduces developers to the code base and explains software design choices. The API Reference describes each class and function in the library. Lastly, we provide a two-page “cheat sheet” designed to be a deskside companion containing examples of common commands in Moco.

The Moco distribution contains examples in Matlab, Python, and C++. These examples range from simple problems, such as predicting the optimum trajectory for a torque-actuated double pendulum, to complex problems, such as predicting a 2-D muscle-driven walking motion. In C++, we provide examples for creating custom cost terms and custom path constraints (TODO). The distribution contains code to generate all the results in this paper.

Future directions

We designed Moco to be easy to use, flexible, and extensible. We verified the software and applied it to multiple musculoskeletal problems. Given this foundation, we expect Moco to accelerate research by reducing the time spent wrangling with simulation tools and enabling our field to tackle more ambitious problems. In its current version, Moco can solve many types of problems: motion prediction, motion tracking, muscle redundancy, and parameter optimization. Moco handles models with kinematic constraints, muscle activation dynamics, compliant contact, and passive force elements; and can minimize a combination of complex costs such as marker tracking and joint reaction loads.

Moco currently lacks the ability to handle certain problems relevant to musculoskeletal biomechanics. Adding support for implicit tendon compliance (i.e., muscle fiber dynamics) will allow researchers to apply Moco to more dynamic movements and to more accurately simulate the behavior of muscles with long tendons (e.g., the triceps surae). Although energy consumption is a commonly desired cost term, direct collocation struggles with the nonconvexity of many energy expenditure models. Koelewijn et al. recently published a smoothed energy consumption model for use in direct collocation~\cite{Koelewijn:2019}; we hope Moco will include this model in the future. Implementing a cost term that minimizes the error between measured and simulated contact forces would improve the accuracy of simulated ground reaction forces in tracking problems. Many direct collocation formulations allow a problem to contain multiple phases, each with different system dynamics. In biomechanics, single stance and double stance could be modeled as separate phases. Multiple phases enables modeling foot–ground contact with kinematic constraints instead of compliant contact, which would avoid the poor numerical conditioning caused by the stiffness of compliant contact. Falisse et al. used a multiple phase approach to solve for muscle parameters that fit multiple unrelated motions. By supporting implicit fiber dynamics, including cost terms for energy expenditure and contact force tracking, and permitting multiple phases, Moco could cover a wider range of biomechanics applications.

The performance and ease of use of Moco’s direct collocation solvers could be improved. Supporting mesh refinement would allow the solver to increase the number of mesh intervals in time ranges with fast dynamics, thereby improving accuracy efficiently (similar to adaptive time stepping). Computing the nonlinear program derivatives with automatic differentiation instead of finite differences would vastly improve the performance of Moco, but would require substantial changes to the Simbody, OpenSim, and Moco source code.

To ensure that Moco will grow to meet the community’s evolving needs, we made the software freely available and we welcome users to contribute to documentation, examples, code, and teaching materials.




\begin{eqnarray}
\label{eq:schemeP}
	\mathrm{P_Y} = \underbrace{H(Y_n) - H(Y_n|\mathbf{V}^{Y}_{n})}_{S_Y} + \underbrace{H(Y_n|\mathbf{V}^{Y}_{n})- H(Y_n|\mathbf{V}^{X,Y}_{n})}_{T_{X\rightarrow Y}},
\end{eqnarray}

\section*{Materials and methods}
\subsection*{Etiam eget sapien nibh}

% For figure citations, please use "Fig" instead of "Figure".
Nulla mi mi, Fig~\ref{fig1} venenatis sed ipsum varius, volutpat euismod diam. Proin rutrum vel massa non gravida. Quisque tempor sem et dignissim rutrum. Lorem ipsum dolor sit amet, consectetur adipiscing elit. Morbi at justo vitae nulla elementum commodo eu id massa. In vitae diam ac augue semper tincidunt eu ut eros. Fusce fringilla erat porttitor lectus cursus, \nameref{S1_Video} vel sagittis arcu lobortis. Aliquam in enim semper, aliquam massa id, cursus neque. Praesent faucibus semper libero.

% Place figure captions after the first paragraph in which they are cited.
\begin{figure}[!h]
\caption{{\bf Bold the figure title.}
Figure caption text here, please use this space for the figure panel descriptions instead of using subfigure commands. A: Lorem ipsum dolor sit amet. B: Consectetur adipiscing elit.}
\label{fig1}
\end{figure}

% Results and Discussion can be combined.
\section*{Results}
Nulla mi mi, venenatis sed ipsum varius, Table~\ref{table1} volutpat euismod diam. Proin rutrum vel massa non gravida. Quisque tempor sem et dignissim rutrum. Lorem ipsum dolor sit amet, consectetur adipiscing elit. Morbi at justo vitae nulla elementum commodo eu id massa. In vitae diam ac augue semper tincidunt eu ut eros. Fusce fringilla erat porttitor lectus cursus, vel sagittis arcu lobortis. Aliquam in enim semper, aliquam massa id, cursus neque. Praesent faucibus semper libero.

% Place tables after the first paragraph in which they are cited.
\begin{table}[!ht]
\begin{adjustwidth}{-2.25in}{0in} % Comment out/remove adjustwidth environment if table fits in text column.
\centering
\caption{
{\bf Table caption Nulla mi mi, venenatis sed ipsum varius, volutpat euismod diam.}}
\begin{tabular}{|l+l|l|l|l|l|l|l|}
\hline
\multicolumn{4}{|l|}{\bf Heading1} & \multicolumn{4}{|l|}{\bf Heading2}\\ \thickhline
$cell1 row1$ & cell2 row 1 & cell3 row 1 & cell4 row 1 & cell5 row 1 & cell6 row 1 & cell7 row 1 & cell8 row 1\\ \hline
$cell1 row2$ & cell2 row 2 & cell3 row 2 & cell4 row 2 & cell5 row 2 & cell6 row 2 & cell7 row 2 & cell8 row 2\\ \hline
$cell1 row3$ & cell2 row 3 & cell3 row 3 & cell4 row 3 & cell5 row 3 & cell6 row 3 & cell7 row 3 & cell8 row 3\\ \hline
\end{tabular}
\begin{flushleft} Table notes Phasellus venenatis, tortor nec vestibulum mattis, massa tortor interdum felis, nec pellentesque metus tortor nec nisl. Ut ornare mauris tellus, vel dapibus arcu suscipit sed.
\end{flushleft}
\label{table1}
\end{adjustwidth}
\end{table}


%PLOS does not support heading levels beyond the 3rd (no 4th level headings).
\subsection*{\lorem\ and \ipsum\ nunc blandit a tortor}
\subsubsection*{3rd level heading} 
Maecenas convallis mauris sit amet sem ultrices gravida. Etiam eget sapien nibh. Sed ac ipsum eget enim egestas ullamcorper nec euismod ligula. Curabitur fringilla pulvinar lectus consectetur pellentesque. Quisque augue sem, tincidunt sit amet feugiat eget, ullamcorper sed velit. Sed non aliquet felis. Lorem ipsum dolor sit amet, consectetur adipiscing elit. Mauris commodo justo ac dui pretium imperdiet. Sed suscipit iaculis mi at feugiat. 

\begin{enumerate}
	\item{react}
	\item{diffuse free particles}
	\item{increment time by dt and go to 1}
\end{enumerate}


\subsection*{Sed ac quam id nisi malesuada congue}

Nulla mi mi, venenatis sed ipsum varius, volutpat euismod diam. Proin rutrum vel massa non gravida. Quisque tempor sem et dignissim rutrum. Lorem ipsum dolor sit amet, consectetur adipiscing elit. Morbi at justo vitae nulla elementum commodo eu id massa. In vitae diam ac augue semper tincidunt eu ut eros. Fusce fringilla erat porttitor lectus cursus, vel sagittis arcu lobortis. Aliquam in enim semper, aliquam massa id, cursus neque. Praesent faucibus semper libero.

\begin{itemize}
	\item First bulleted item.
	\item Second bulleted item.
	\item Third bulleted item.
\end{itemize}

\section*{Discussion}
Nulla mi mi, venenatis sed ipsum varius, Table~\ref{table1} volutpat euismod diam. Proin rutrum vel massa non gravida. Quisque tempor sem et dignissim rutrum. Lorem ipsum dolor sit amet, consectetur adipiscing elit. Morbi at justo vitae nulla elementum commodo eu id massa. In vitae diam ac augue semper tincidunt eu ut eros. Fusce fringilla erat porttitor lectus cursus, vel sagittis arcu lobortis. Aliquam in enim semper, aliquam massa id, cursus neque. Praesent faucibus semper libero~\cite{bib3}.

\section*{Conclusion}

CO\textsubscript{2} Maecenas convallis mauris sit amet sem ultrices gravida. Etiam eget sapien nibh. Sed ac ipsum eget enim egestas ullamcorper nec euismod ligula. Curabitur fringilla pulvinar lectus consectetur pellentesque. Quisque augue sem, tincidunt sit amet feugiat eget, ullamcorper sed velit. 

Sed non aliquet felis. Lorem ipsum dolor sit amet, consectetur adipiscing elit. Mauris commodo justo ac dui pretium imperdiet. Sed suscipit iaculis mi at feugiat. Ut neque ipsum, luctus id lacus ut, laoreet scelerisque urna. Phasellus venenatis, tortor nec vestibulum mattis, massa tortor interdum felis, nec pellentesque metus tortor nec nisl. Ut ornare mauris tellus, vel dapibus arcu suscipit sed. Nam condimentum sem eget mollis euismod. Nullam dui urna, gravida venenatis dui et, tincidunt sodales ex. Nunc est dui, sodales sed mauris nec, auctor sagittis leo. Aliquam tincidunt, ex in facilisis elementum, libero lectus luctus est, non vulputate nisl augue at dolor. For more information, see \nameref{S1_Appendix}.

\section*{Supporting information}

% Include only the SI item label in the paragraph heading. Use the \nameref{label} command to cite SI items in the text.
\paragraph*{S1 Fig.}
\label{S1_Fig}
{\bf Bold the title sentence.} Add descriptive text after the title of the item (optional).

\paragraph*{S2 Fig.}
\label{S2_Fig}
{\bf Lorem ipsum.} Maecenas convallis mauris sit amet sem ultrices gravida. Etiam eget sapien nibh. Sed ac ipsum eget enim egestas ullamcorper nec euismod ligula. Curabitur fringilla pulvinar lectus consectetur pellentesque.

\paragraph*{S1 File.}
\label{S1_File}
{\bf Lorem ipsum.}  Maecenas convallis mauris sit amet sem ultrices gravida. Etiam eget sapien nibh. Sed ac ipsum eget enim egestas ullamcorper nec euismod ligula. Curabitur fringilla pulvinar lectus consectetur pellentesque.

\paragraph*{S1 Video.}
\label{S1_Video}
{\bf Lorem ipsum.}  Maecenas convallis mauris sit amet sem ultrices gravida. Etiam eget sapien nibh. Sed ac ipsum eget enim egestas ullamcorper nec euismod ligula. Curabitur fringilla pulvinar lectus consectetur pellentesque.

\paragraph*{S1 Appendix.}
\label{S1_Appendix}
{\bf Lorem ipsum.} Maecenas convallis mauris sit amet sem ultrices gravida. Etiam eget sapien nibh. Sed ac ipsum eget enim egestas ullamcorper nec euismod ligula. Curabitur fringilla pulvinar lectus consectetur pellentesque.

\paragraph*{S1 Table.}
\label{S1_Table}
{\bf Lorem ipsum.} Maecenas convallis mauris sit amet sem ultrices gravida. Etiam eget sapien nibh. Sed ac ipsum eget enim egestas ullamcorper nec euismod ligula. Curabitur fringilla pulvinar lectus consectetur pellentesque.

\section*{Acknowledgments}
Cras egestas velit mauris, eu mollis turpis pellentesque sit amet. Interdum et malesuada fames ac ante ipsum primis in faucibus. Nam id pretium nisi. Sed ac quam id nisi malesuada congue. Sed interdum aliquet augue, at pellentesque quam rhoncus vitae.

\nolinenumbers

% Either type in your references using
% \begin{thebibliography}{}
% \bibitem{}
% Text
% \end{thebibliography}
%
% or
%
% Compile your BiBTeX database using our plos2015.bst
% style file and paste the contents of your .bbl file
% here. See http://journals.plos.org/plosone/s/latex for 
% step-by-step instructions.
%

\bibliography{MocoPaper.bib}

%\begin{thebibliography}{10}
%
%\bibitem{bib1}
%Conant GC, Wolfe KH.
%\newblock {{T}urning a hobby into a job: how duplicated genes find new
%  functions}.
%\newblock Nat Rev Genet. 2008 Dec;9(12):938--950.
%
%\bibitem{bib2}
%Ohno S.
%\newblock Evolution by gene duplication.
%\newblock London: George Alien \& Unwin Ltd. Berlin, Heidelberg and New York:
%  Springer-Verlag.; 1970.
%
%\bibitem{bib3}
%Magwire MM, Bayer F, Webster CL, Cao C, Jiggins FM.
%\newblock {{S}uccessive increases in the resistance of {D}rosophila to viral
%  infection through a transposon insertion followed by a {D}uplication}.
%\newblock PLoS Genet. 2011 Oct;7(10):e1002337.
%
%\end{thebibliography}



\end{document}

